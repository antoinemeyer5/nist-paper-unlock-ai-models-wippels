\section{Introduction}
\label{sec:intro}

With the relentless development of Artificial Intelligence (AI), new
architectures and new ways of analyzing texts and images are leading to many new
models. This enthusiasm enables us to meet a wide range of needs but analyzing
all these tools and choosing the right AI model to best solve your specific
problem is becoming less and less straightforward.

Building a model from scratch for a specific application is time-consuming and
tedious. Numerous public repositories of AI models exist, avoiding this
preliminary work and allowing you to concentrate on use (inference) and/or
adjustments (finetuning). It is therefore vital to be able to select the most
accurate pre-trained model from those available, if possible, as automatically
as possible.

This motivation of reusing pre-trained models from public repositories poses
three major challenges: (1) there is no standard API to access public
repositories with trained AI models, (2) there is insufficient metadata about
trained AI models in public repositories to match them to applications and to
run them, and (3) there is no way of automatically assessing the accuracy of AI
models.

The first challenge is partly facilitated by the Application Programming
Interface (API) offered by certain platforms, such as Transformers for Hugging
Face. However, this is not the case for all AI model repositories and Web Image
Processing Pipeline (WIPP) has yet to implement these APIs. The second challenge
also has a start of a solution in the shape of the Model Card concept. However,
information is often scanty if not completely absent and is not automated.
Finally, the third challenge requires special development to enable calculation
of the Dice-Sørensen coefficient.
