\section{Results}
\label{sec:results}

Access new repositories give access to a lot of new models usable in Web Image
Processing Pipeline (WIPP). It also saves hours of computational time and
resources, as well as improves reusability of pre-trained AI models.

In the context of Image Analysis in WIPP, we are looking to use segmentation
mask models.

\begin{table}[H]
\centering
\caption{\label{tab:number_of_newly_available_models}%
  Number of newly available image segmentation mask models thanks to Inference Plugin
}
\begin{tabular}{lc}
  \toprule
  Repository & Mask Generation Model(s) \\
  \midrule
  WIPP & 1 \\
  Hugging Face & 176 \\
  BioImage.IO & 32 \\
  Cellpose & 21 \\
  SAM2 & 8 \\
  \bottomrule
\end{tabular}
\end{table}

The level of documentation varies greatly from model to model. We have set up an
automatic documentation system for models trained within WIPP.

Finally to avoid lower confidence in external models, we have developed a way of
evaluating the results and therefore the relevance of the models tested, in
order to select and retain only those that deliver the best results.

We have selected 5 models, one from each previous repositories.

\begin{table}[H]
\centering
\begin{tabular}{lll}
  \toprule
  Repository & Model Name & ID \\
  \midrule
  WIPP & UNet CNN 1.0.0 & U \\
  Hugging Face & Zigeng/SlimSAM-uniform-50 & H \\
  BioImage.IO & 10.5281/zenodo.5869899 & B \\
  Cellpose & cyto3 & C \\
  SAM2 & facebook/sam2.1-hiera-large & S \\
  \bottomrule
\end{tabular}
\end{table}

We compute everything in WIPP. The WIPP server specifications are:
\begin{itemize}
  \item CPU: AMD Ryzen 9 3950X 16-Core Processor with 16 cores and 2 threads
  \item GPU: NVIDIA GeForce RTX 3090
  \item 64G RAM
\end{itemize}

We compute results on two different datasets: 3D-RPE Week 1 and 3D-RPE Week 3.

\subsection{Dataset 3D-RPE Week 1}

%Masks used are here
%https://wipp-dev.nist.gov/images-collection/671abb516103440e64a77397 and images
%used are here
%/mnt/isgnas/project/csmet/bio/Proj-028-RPE-Measurements/sample\_data\_2024-10-24.
%There are 162 images.

We make a comparative table where we compute the results for different Ground
Truth (GT). The table can be read as: the value in column X and row Y, you have
the accuracy of the model X based on the ground truth of Y. In this part the
dataset is always 3D-RPE Week 1 which we will simply call W1.

Tables are just half filled because the value in column X and row Y is the
same as the value in column Y and row X.

\begin{table}[H]
\small
\centering
\caption{\label{tab:base3dRPEdatamask}%
  Accuracy (\%) with different Ground Truth (GT)
}
\begin{tabular}{lccccc}
  \toprule
  GT & Model U & Model H & B & C & S \\
  \midrule
  W1 &  & \makecell{$\SI{25.05}{\percent}$ \\ $\pm \SI{22.02}{\percent}$} &  &  & \makecell{$\SI{38.16}{\percent}$ \\ $\pm \SI{35.02}{\percent}$} \\
  U & x &  &  &  &  \\
  H & x & x &  &  & \makecell{$\SI{82.51}{\percent}$ \\ $\pm \SI{18.29}{\percent}$} \\
  B & x & x & x &  &  \\
  C & x & x & x & x &  \\
  S & x & x & x & x & x \\
  \bottomrule
\end{tabular}
\end{table}

\subsection{Dataset 3D-RPE Week 3}

%Masks used are here \TODO\ and images used are here \TODO\ .
%There are 162 images.

In this part the dataset is always 3D-RPE Week 3 which we will simply call W3.

\begin{table}[H]
\small
\centering
\caption{\label{tab:base3dRPEdatamask}%
  Accuracy (\%) with different Ground Truth (GT)
}
\begin{tabular}{lccccc}
  \toprule
  GT & Model U & Model H & B & C & S \\
  \midrule
  W3 &  & \makecell{$\SI{91.87}{\percent}$ \\ $\pm \SI{7.11}{\percent}$} &  &  & \makecell{$\SI{77.69}{\percent}$ \\ $\pm \SI{26.12}{\percent}$} \\
  U & x &  &  &  &  \\
  H & x & x &  &  & \makecell{$\SI{89.42}{\percent}$ \\ $\pm \SI{15.18}{\percent}$} \\
  B & x & x & x &  &  \\
  C & x & x & x & x &  \\
  S & x & x & x & x &  x \\
  \bottomrule
\end{tabular}
\end{table}


%\TODO\

%ajm32@pn125927:/mnt/isgnas/project/csmet/bio/Proj-028-RPE-Measurements/NEI\_RPE\_GoogleMap/Final\_stacks/TOM20/P1-W3-TOM

%\TODO\

\subsection{Notes}

Even if the results may not be perfect, this method allows you to quickly try
out a new model at reduced cost. It is also easy to change dataset. We can then
take a promising model and improve it by finetuning it. We hope to improve the
speed of data analysis within WIPP and enable better overall results.
