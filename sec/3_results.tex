\section{Results}
\label{sec:results}

Access new repositories give access to a lot of new models usable in Web Image
Processing Pipeline (WIPP). It also saves hours of computational time and
resources, as well as improves reusability of pre-trained AI models.

In the context of Image Analysis in WIPP, we are looking to use Mask
Generation Models.

\begin{table}[H]
\centering
\caption{\label{tab:number_of_newly_available_models}%
  Number of newly available Mask Generation models
}
\begin{tabular}{lc}
  \toprule
  Repository & Mask Generation Model(s) \\
  \midrule
  WIPP & 1 \\
  Hugging Face & 176 \\
  BioImage.IO & 32 \\
  Cellpose & 21 \\
  SAM2 & 8 \\
  \bottomrule
\end{tabular}
\end{table}

The level of documentation varies greatly from model to model. We have set up an
automatic documentation system for models trained within WIPP.

Finally to avoid lower confidence in external models, we have developed a way of
evaluating the results and therefore the relevance of the models tested, in
order to select and retain only those that deliver the best results.

\subsection{WIPP and 3D-RPE}

Masks used are here
https://wipp-dev.nist.gov/images-collection/671abb516103440e64a77397 and images
used are here
/mnt/isgnas/project/csmet/bio/Proj-028-RPE-Measurements/sample\_data\_2024-10-24.
The WIPP server specifications are \TODO\

\begin{table}[H]
\tiny
\centering
\caption{\label{tab:comparative_results_for_different_models}%
  Comparative models results run in WIPP
}
\begin{tabular}{llcc}
  \toprule
  Repository & Model Name & Inference Time per Image (s) & Accuracy (\%) \\
  \midrule
  WIPP & WIPP UNet CNN 1.0.0 & \TODO\ &  \\
  Hugging Face & facebook/sam-vit-huge & 7.04 & $\SI{40.62}{\percent} \pm \SI{34.99}{\percent}$ \\
               & Zigeng/SlimSAM-uniform-50 & 939s/162 images = 5.79 & $\SI{25.05}{\percent} \pm \SI{22.02}{\percent}$ \\
               & jadechoghari/robustsam-vit-large & 6.76 & $\SI{38.19}{\percent} \pm \SI{35.67}{\percent}$ \\
  BioImage.IO & 10.5281/zenodo.5869899 & ISSUE WITH DOCKER &  \\
              & 10.5281/zenodo.5764892 &  &  \\
  Cellpose & cyto3 & \TODO\ &  \\
           & nuclei & \TODO\ &  \\
  SAM2 & facebook/sam2.1-hiera-large & 2.12 & $\SI{38.16}{\percent} \pm \SI{35.02}{\percent}$ \\
       & facebook/sam2-hiera-small & \TODO\ &  \\
  \bottomrule
\end{tabular}
\end{table}

\subsection{Local and 3D-RPE}

Data used are the same from above. The local computer specifications are one GPU
Quadro RTX 4000 with CUDA version 12.7 and Memory 8192 MiB.

\begin{table}[H]
\tiny
\centering
\caption{\label{tab:comparative_results_for_different_models}%
Comparative models results run in local
}
\begin{tabular}{llcc}
  \toprule
  Repository & Model Name & Inference Time per Image (s) & Accuracy (\%) \\
  \midrule
  Hugging Face & facebook/sam-vit-huge & \TODO\ &  \\
               & Zigeng/SlimSAM-uniform-50 & &  \\
               & jadechoghari/robustsam-vit-large &  &  \\
  BioImage.IO & 10.5281/zenodo.5869899 & \TODO\ &  \\
              & 10.5281/zenodo.5764892 &  &  \\
  Cellpose & cyto3 & \TODO\ &  \\
           & nuclei &  &  \\
  \bottomrule
\end{tabular}
\end{table}

\subsection{Local and RPEimplants}

Data used are from this link https://isg.nist.gov/deepzoomweb/data/RPEimplants.
They contains images of 2D Measurement of Retinal Pigment Epithelium Function
Using Quantitative Bright-Field Microscopy. Local computer specifications are
the same as above.

\begin{table}[H]
\tiny
\centering
\begin{tabular}{llcc}
  \toprule
  Repository & Model Name & Inf. Time per Image (s) & Accuracy (\%) \\
  \midrule
  Hugging Face & facebook/sam-vit-huge & 4.16 & $\SI{85.87}{\percent} \pm \SI{3.98}{\percent}$ \\
                & Zigeng/SlimSAM-uniform-50 & 2.81 & $\SI{79.78}{\percent} \pm \SI{5.31}{\percent}$ \\
                & jadechoghari/robustsam-vit-large &  &  \\
  BioImage.IO & 10.5281/zenodo.5869899 & 0.31 & $\SI{89.30}{\percent} \pm \SI{0.84}{\percent}$ \\
              & 10.5281/zenodo.5764892 & 0.38 & $\SI{10.44}{\percent} \pm \SI{3.20}{\percent}$ \\
  Cellpose & cyto3 & \TODO\ &  \\
            & nuclei &  &  \\
  \bottomrule
\end{tabular}
\end{table}

---

Even if the results may not be perfect, this method allows you to quickly try
out a new model at reduced cost. We can then take a promising model and improve
it by finetuning it. We hope to improve the speed of data analysis within WIPP
and enable better overall results.
