\section{Results}
\label{sec:results}

Access new repositories give access to a lot of new models usable in Web Image
Processing Pipeline (WIPP). It also saves hours of computational time and
resources, as well as improves reusability of pre-trained AI models.

In the context of Image Analysis in WIPP, we are looking to use segmentation
mask models.

\begin{table}[H]
\centering
\caption{\label{tab:number_of_newly_available_models}%
  Number of newly available image segmentation mask models thanks to Inference Plugin
}
\begin{tabular}{lc}
  \toprule
  Repository & Mask Generation Model(s) \\
  \midrule
  WIPP & 1 \\
  Hugging Face & 176 \\
  BioImage.IO & 32 \\
  Cellpose & 21 \\
  SAM2 & 8 \\
  \bottomrule
\end{tabular}
\end{table}

The level of documentation varies greatly from model to model. We have set up an
automatic documentation system for models trained within WIPP.

Finally to avoid lower confidence in external models, we have developed a way of
evaluating the results and therefore the relevance of the models tested, in
order to select and retain only those that deliver the best results.

We have selected 10 models from previous repositories.

\begin{table}[H]
\centering
\begin{tabular}{lll}
  \toprule
  Repository & Model Name & ID \\
  \midrule
  WIPP & WIPP UNet CNN 1.0.0 & W1 \\
  Hugging Face & facebook/sam-vit-huge & H1 \\
                & Zigeng/SlimSAM-uniform-50 & H2 \\
                & jadechoghari/robustsam-vit-large & H3 \\
  BioImage.IO & 10.5281/zenodo.5869899 & B1 \\
              & 10.5281/zenodo.5764892 & B2 \\
  Cellpose & cyto3 & C1 \\
            & nuclei & C2 \\
  SAM2 & facebook/sam2.1-hiera-large & S1 \\
        & facebook/sam2-hiera-small & S2 \\
  \bottomrule
\end{tabular}
\end{table}

We compute everything in WIPP. The WIPP server specifications are \TODO\

\subsection{Dataset 3D-RPE data week 1}

Masks used are here
https://wipp-dev.nist.gov/images-collection/671abb516103440e64a77397 and images
used are here
/mnt/isgnas/project/csmet/bio/Proj-028-RPE-Measurements/sample\_data\_2024-10-24.
There are 162 images.

\begin{table}[H]
\centering
\caption{\label{tab:base3dRPEdatamask}%
  Compute time per image (s) and accuracy (\%) with 3D-RPE data week 1 masks as ground truth
}
\begin{tabular}{lll}
  \toprule
  ID & Time & Accuracy \\
  \midrule
  W1 & \TODO\ &  \\
  H1 & 7.04 & $\SI{40.62}{\percent} \pm \SI{34.99}{\percent}$ \\
  H2 & 5.79 & $\SI{25.05}{\percent} \pm \SI{22.02}{\percent}$ \\
  H3 & 6.76 & $\SI{38.19}{\percent} \pm \SI{35.67}{\percent}$ \\
  B1 & \TODO\ & \TODO\ \\
  B2 & \TODO\ & \TODO\ \\
  C1 & \TODO\ & \TODO\ \\
  C2 & \TODO\ & \TODO\ \\
  S1 & 2.12 & $\SI{38.16}{\percent} \pm \SI{35.02}{\percent}$ \\
  S2 & \TODO\ &  \\
  \bottomrule
\end{tabular}
\end{table}

\begin{table}[H]
\centering
\caption{\label{tab:baseW1}%
  Compute time per image (s) and accuracy (\%) with W1 results as ground truth
}
\begin{tabular}{lll}
  \toprule
  ID & Time & Accuracy \\
  \midrule
  W1 & / & $\SI{100.0}{\percent} \pm \SI{0.0}{\percent}$ \\
  H1 & \TODO\ & \TODO\ \\
  H2 & \TODO\ & \TODO\ \\
  H3 & \TODO\ & \TODO\ \\
  B1 & \TODO\ & \TODO\ \\
  B2 & \TODO\ & \TODO\ \\
  C1 & \TODO\ & \TODO\ \\
  C2 & \TODO\ & \TODO\ \\
  S1 & \TODO\ & \TODO\ \\
  S2 & \TODO\ & \TODO\ \\
  \bottomrule
\end{tabular}
\end{table}

\begin{table}[H]
\centering
\caption{\label{tab:baseW1}%
  Compute time per image (s) and accuracy (\%) with H1 results as ground truth
}
\begin{tabular}{lll}
  \toprule
  ID & Time & Accuracy \\
  \midrule
  W1 & \TODO\ & \TODO\ \\
  H1 & / & $\SI{100.0}{\percent} \pm \SI{0.0}{\percent}$ \\
  H2 & 5.79 & $\SI{78.60}{\percent} \pm \SI{17.41}{\percent}$ \\
  H3 & 6.76 & $\SI{93.70}{\percent} \pm \SI{13.32}{\percent}$ \\
  B1 & \TODO\ & \TODO\ \\
  B2 & \TODO\ & \TODO\ \\
  C1 & \TODO\ & \TODO\ \\
  C2 & \TODO\ & \TODO\ \\
  S1 & 2.12 & $\SI{93.37}{\percent} \pm \SI{12.52}{\percent}$ \\
  S2 & \TODO\ & \TODO\ \\
  \bottomrule
\end{tabular}
\end{table}

\subsection{Dataset 3D-RPE data week 3}

Masks used are here \TODO\ and images used are here \TODO\. There are 162 images.

\begin{table}[H]
\centering
\caption{\label{tab:base3dRPEdatamask-2}%
  Compute time per image (s) and accuracy (\%) with 3D-RPE data week 3 masks as ground truth
}
\begin{tabular}{lll}
  \toprule
  ID & Time & Accuracy \\
  \midrule
  W1 & \TODO\ & \TODO\ \\
  H1 & \TODO\ & \TODO\ \\
  H2 & \TODO\ & \TODO\ \\
  H3 & \TODO\ & \TODO\ \\
  B1 & \TODO\ & \TODO\ \\
  B2 & \TODO\ & \TODO\ \\
  C1 & \TODO\ & \TODO\ \\
  C2 & \TODO\ & \TODO\ \\
  S1 & \TODO\ & \TODO\ \\
  S2 & \TODO\ & \TODO\ \\
  \bottomrule
\end{tabular}
\end{table}









ajm32@pn125927:/mnt/isgnas/project/csmet/bio/Proj-028-RPE-Measurements/NEI\_RPE\_GoogleMap/Final\_stacks/TOM20/P1-W3-TOM

\TODO\









\subsection{Local compute with RPEimplants data}

Data used are from this link https://isg.nist.gov/deepzoomweb/data/RPEimplants.
They contains images of 2D Measurement of Retinal Pigment Epithelium Function
Using Quantitative Bright-Field Microscopy. There are 1032 images but for this
results we compute on only 210 images. The local computer specifications are one
GPU Quadro RTX 4000 with CUDA version 12.7 and Memory 8192 MiB.

\begin{table}[H]
\centering
\caption{\label{tab:baseMask}%
  Compute time per image (s) and accuracy (\%) with given masks as ground truth
}
\begin{tabular}{lll}
  \toprule
  ID & Time & Accuracy \\
  \midrule
  H1 & 4.16 & $\SI{85.87}{\percent} \pm \SI{3.98}{\percent}$ \\
  H2 & 2.81 & $\SI{79.78}{\percent} \pm \SI{5.31}{\percent}$ \\
  H3 & 3.57 & $\SI{83.08}{\percent} \pm \SI{3.34}{\percent}$ \\
  B1 & 0.31 & $\SI{89.30}{\percent} \pm \SI{0.84}{\percent}$ \\
  B2 & 0.38 & $\SI{10.44}{\percent} \pm \SI{3.20}{\percent}$ \\
  \bottomrule
\end{tabular}
\end{table}

\subsection{Notes}

Even if the results may not be perfect, this method allows you to quickly try
out a new model at reduced cost. It is also easy to change dataset. We can then
take a promising model and improve it by finetuning it. We hope to improve the
speed of data analysis within WIPP and enable better overall results.
